\section{Future Work}
\label{discussion}
Due to limitations of available resources, there is 
more potential work to be done in evaluating D-Sync
It would be interesting to evaluate our system over different
network conditions. Our WAN evalaution was limited because
we had only 2 machines that we were able to install our 
software on and therefore were only able to use 2 different
LAN's at a time. Similarly, it would be interesting to 
evaluate the performance of Dropbox over different
working group sizes to see how it scales with the number 
of users and compare that to our own results.

There are also many things that could be done to improve D-Sync.
Although security of files was a significant motivator for our
project, we focused on making a system that would be easy to 
make secure, but did not implement any encryption scheme, 
as this was beyond our scope. Another necessary inclusion would
be conflict resolution for files with no unambiguous 'most recent'
version. This could be done by either implementing some sort of
version control system, requiring users to resolve conflicts in
real time, or creating conflict files that users can resolve
asynchronously. Again, while necessary for a user-friendly 
system, this was beyond our scope. Once conflict resolution and/or
file encryption have been added to D-Sync, it would be 
interesting to evaluate what effect, if any, they have on 
the system's overall performance.

Lastly, while our system uses a distributed storage scheme,
communication is controlled in a undesirably-centralized fashion.
Luckily, as described in \ref{design.distributed}, a distributed
version of our broker system could be implemented and, provided 
it implements the same API as our more-centralized broker, could
be easily swapped in. This process would be completely transparent
to the client, with the exception of any changes to performance.
In this way, we have layed the groundwork and created a easily-extendible
framework for any future improvements to the broker protocol. We have
included the current version of our code, as of this publication, and
the most recent version will continue to be maintained on our
Github repository: \emph{https://github.com/dzstevens/CIS-630}
