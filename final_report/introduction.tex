\section{Introduction}
\label{introduction}
% Introduction to file hosting systems
Of the wide range of today's Internet tools,
file hosting systems such as Dropbox and Google Drive are certainly among the most popular.
These systems offer convenient (and often cheap) means
to store and backup data, in addition to making that data available
to any device with Internet access and the proper credentials.
The ever-growing ubiquity of Internet access
and the advent of cloud storage have helped
these systems flourish,
allowing them to serve billions of files for millions of users.

As the most popular file hosting system on today's Internet,
Dropbox leads the way in this growing domain.
Since January 2010, Dropbox has steadily expanded from about 4 million to 100 million users,
and increased its usage rate from roughly 250 million to 1 billion files saved per day~\cite{dropboxstats}.
Stop for a moment and consider the sheer magnitude of these statistics:
even roughly averaged, this means that Dropbox users collectively save over
1 million files \emph{every two minutes}.

This frequency of usage points to a key aspect of Dropbox's popularity:
its capacity for collaborative work.
The system allows its users to share data with other select users
in the so-called working group, and also keep a local copy
of the shared data on any owned machine or device.
Most importantly, Dropbox synchronizes quickly and automatically across local and cloud replicas
by employing \textbf{delta encoding},
in which only the \emph{changes} to files are saved and uploaded.
The efficiency of this synchronization strategy makes Dropbox a convenient tool
for both synchronous and asynchronous collaboration among working groups.
The benefits of these features have not gone
unnoticed, and continue to attract attention~\cite{dropboxmercurial,dropboxcollab}.

% Privacy problem with file hosting systems
When we consider data privacy in Dropbox and other cloud-based file hosting systems, however, we see that not everything is silver lining.
Privacy issues with cloud services have recently
fallen under scrutiny by the security community~\cite{van2010impossibility,soghoian2010caught}, and Dropbox has not escaped this critical eye.
Aside from a few exposed security issues~\cite{mulazzani2011dark,dropboxsecurity},
the top file hosting system has been criticized for its policies on data discretion.
In order to facilitate delta encoding,
Dropbox maintains encryption keys to calculate hashes for saved files. When a change is detected in a shared folder,
Dropbox hashes blocks of the data and uses those hashes to see if they are stored \emph{anywhere} on \emph{any} of its servers.
Using this information, it only needs to upload the new chunks. This seemingly clever trick can enable a gigabyte file to almost 'magically'
upload in seconds. It's also a sweet deal for the company, as it allows them to offer more space to clients at almost no cost if the client is sharing
data shared somewhere else. 

Unfortunately, this optimization comes at the cost of security. One simple attack is immediately obvious: 
Imagine an agent, Alice, that wants to know if a given copy-writed media file has been shared on Dropbox.
She could simply create a Dropbox account, add the file to her shared folder and watch to see how quickly it uploads. 
If it uploaded far too quickly, she immediately knows that it has already been uploaded and hashed somewhere else.
She could then obtain a subpoena to require Dropbox to disclose which users have this file. Since Dropbox does not 
let users control how their files are encrypted, and since they have already hashed the location of every block of this file,
they can easily determine who has these files and would be required by law to report them.~\cite{dropboxprivacy}.

One solution to the privacy issues of centralized data would be to distribute
data storage across users' local machines.
The file hosting system itself would remain blind to this data, 
and merely marshall encrypted updates between user machines to handle synchronization.
For such a distributed file hosting system to be successful, it would have to achieve
the following design goals (in addition to privacy):
\begin{itemize}
    \item \textbf{Accuracy:} The system must be able to properly synchronize all updates across all \emph{online} users. In other words, whenever a user needs to pull
    data from the system, it must pull a completely up-to-date copy.  Furthermore, a user should not push outdated data to the system.
    \item \textbf{Transparency:} The system itself should largely be transparent to the user. 
    To facilitate transparency, the system must enable automated and asynchronous updates for shared data. When a user is online,
    changes should be pushed immediately after they occur. In addition, a user should not have to block and wait for the system to update.
    \item \textbf{Scalability:} The system must be scalable with respect to increasing workloads and working group sizes.
    \item \textbf{Efficiency:} The system must be efficient with respect to the latency of message and data transfers.
    In particular, the amount of extra information (beyond file data) 
    that is passed between networked entities should be small.
\end{itemize}

% Potential Solution: Distributed file hosting system
In this work, we present D-Sync, a distributed and automatically synchronizing file hosting
architecture that addresses the privacy issues of centralized data storage.
D-Sync addresses these concerns
by distributing data in workspace directories across user machines, called \textbf{clients}.
Clients make changes to the files within their local workspace,
and changes are automatically pushed to the system.
A \textbf{broker} program marshalls these changes and ensures that all online local workspaces are updated,
but remains blind to the actual data itself.
In this manner, working groups can collaborate efficiently over the Internet
without the privacy concerns of centralized data.

% How we evaluate the system
We evaluate our system's performance with respect to the latency
of synchronization across different networking scenarios.
In particular, we measure the total time it takes to synchronize
all clients with variable working group and update file size.
We also benchmark our system against Dropbox's performance.
Our comparative analysis shows that the D-Sync architecture
scales reasonably with larger working groups and workloads,
and furthermore outperforms Dropbox in several scenarios.

The contributions of this work include:
\begin{itemize}
\item \emph{D-Sync: a distributed file hosting architecture.} 
automated synchronization of files and protects the privacy of its users.
Our architecture includes schematics for its major components (clients and brokers) as well as a communication protocol and automated synchronization strategy.
While similar proposals exist, to the best of our knowledge 
our work is the first to posit such an architecture as a privacy-preserving alternative to conventional file hosting systems.
\item \emph{An evaluation of distributed file hosting architectures.}
In evaluating our prototype system's performance and accuracy,
we provide a first look at how such architectures may perform
on various network scenarios.
Furthermore, we conduct a comparative analysis of Dropbox's
performance, and present the results in conjunction with our prototype evaluation.
\end{itemize}

The remainder of our paper is organized as follows.
In Section~\ref{related} we highlight previous work related to our project.
Section~\ref{design} covers our system design, including
its major components and design decisions,
and Section~\ref{prototype} details a prototype we built using this design.
We evaluate our system in Section~\ref{evaluation}, specifically
analyzing its performance and synchronization correctness.
In Section~\ref{discussion} we discuss our work and future directions,
and finally we conclude in Section~\ref{conclusion}.
Section~\ref{appendix} contains a brief description
of our code and division of labor on this project.
