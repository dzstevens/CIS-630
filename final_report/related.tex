\section{Related Work}
\label{related}
The topics of synchronization and replication strategies have been considerably researched
within the general domain of distributed file systems.
Much of this work has focused on replica maintenance for the purpose of data availability~\cite{pu1991replica,damani1999optimistic,goel2006data,chun2006replica,ford2010availability}.
Pu et al's work discusses various replica control mechanisms for maintaining epsilon-copy serializability (ESR), 
a correctness criterion that allows asynchronous maintenance of mutual consistency for replicas~\cite{pu1991replica}. 
However, their evaluation does not extend to specific domains of distributed file systems.
Their research also lays valuable foundations for evaluating replication strategies in specific domains (\ie~\cite{ford2010availability}). 
While we also build on Pu et al's groundwork, our emphasis on automated synchronization and privacy constitute a distinct focus from other follow up research.

In another similar work, Chun et al evaluate replication strategies for storage systems distributed over the Internet~\cite{chun2006replica}.
Their research focuses on data durability, the assurance that data put into the system is not lost due to disk failures, and claims that this property be held more important than conventional availability.
While their work uncovers a valuable property that should be widely applicable in distributed storage systems,
our research focuses specifically on availability as a necessary prerequisite of automated synchronization.

Lastly, a couple works on distributed collaboration systems closely mirror that of our own~\cite{oster2006data,merle2012decentralized}.
These works discuss peer-to-peer based architectures for distributed collaboration tools,
including proposed system designs built atop existing architectures (GitHub) and
provably correct synchronization models.
[FILL IN why different]
\comment{
Rather than propose a specific design or   
    %p2p models such as:
        %Merle:
        %    focus: centralized collab->distributed collaboration tools
        %    idea: eliminate central entity, use p2p architecture defined by working group
        %    method: build structure on top of git-hub
        %    difference: focus on proposing a system, 
        %                whereas we want to evaluate a methodology
        %                built on github, which although worthy, has its own problems
        %Oster:
        %    focus: centralized collab tools -> p2p
        %    idea: WOOT framework
        %    method: declare methodology for 
}

\comment{
    Ford: availability in a cloud storage system given different system parameters.
    NOT decentralized storage environment.
    %TODO find more, also check out methodology

    Chun: replica algorithms should 1) focus on durability, a less expensive and more useful goal than availability; 2) durability algorithms must create new copies of data faster than failures destroy them; and 3) more replicas helps tolerate bursts of failures but not increasing likelihood of disk failure.
    Chun focuses on durability, but we need high availability due to need to keep data in local repository current.

    Pu: Discusses various replica control mechanisms which use epsilon-copy serializability (ESR), a correctness criterion that allows asynchronous maintenance of mutual consistency of replicated data. ESR allows inconsistent data to be read, but requires data to eventually converge to 1SR state.
    "Standard correctness criterion for coherency control such as 1-copy serializability (1SR) are hard to attain with asynchronous coherency control."
    Pu describes and analyzes these various replica control methods, but does not evaluate them in a specific use case such as [file hosting | revision] systems.
}

