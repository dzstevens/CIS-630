\section{Activities}
%address the timeline:
  %\begin{itemize}
  %\item \bf Week ~3 -- 4  \rm~Literature review and familiarization with GraphLab
  %\item \bf Weeks 5 -- 7  \rm~Design and implementation
  %\item \bf Weeks 8 -- 9  \rm~Experimentation and evaluation
  %\item \bf Week ~10~~~ \rm~Paper and presentations
  %\end{itemize}
%What have we done so far?
In this section, we cover the work we have done in the first half of the term.
At this point in the project, we are on schedule with respect to our timeline.
We began with focused literature review.
We first reviewed machine learning inference algorithms including Markov logic networks
and belief propagation, in order to understand the general procedure for cleaning
up a noisy knowledge base.
We used Shangpu Jiang's paper submission on refining NELL's~\cite{carlson2010toward} knowledge base with the Markov Chain Monte Carlo algorithm as reference,
as well as web resources and conversations with Shangpu and Professor Daniel Lowd.

In addition to literature review, we also began familiarizing ourselves with GraphLab~\cite{low2010graphlab}.
We downloaded the GraphLab repository to our local machines, and compiled and built the framework.
We then investigated GraphLab's graphical models toolkit, 
which provides the parallel loopy belief propagation algorithm for refining structured noisy data such as NELL's knowledge base.
We examined the required source file formats and analyzed the toolkit's source code for the inference algorithm,
and finally ran a successful test on a sample dataset.

After working through the literature review and the GraphLab framework,
we began our design and implementation phase on schedule in week five.
We obtained our dataset of NELL's extracted knowledge base,
which was then preprocessed to ground all first-order logic clauses
into independent variables.
Next, we built a Python script to parse the corrected dataset
and convert it into the vertex and edge datasets (beliefs and relations between beliefs, respectively) necessary for the graphical models toolkit.
Finally, we ran a successful initial test on a single machine, and observed the results.

In the following section, we detail the issues and findings we encountered during the first half of our work.
